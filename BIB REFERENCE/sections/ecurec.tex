\textbf{Recurrencia Lineal Homog\'enea:}
\begin{itemize}
    \item{
		La forma general es $a_0 t_n+ a_1t_{n {−} 1} + \cdots + a_k t_{{n} {− k}} = 0$
    }
    \item{
		Su ecuaci\'on caracter\'istica es $a_0 x^k +a_1 x^{{k}{−1}} + \cdots + a^k = 0$
    }
    \item{
		 Suponiendo que $r_1, r_2, r_3, ..,r_k$ son $k$ raices distintas de la ecuaci\'on 
		 caracter\'istica, la soluci\'on ser\'a $t_n = \sum_{i=1}^{k} c_i r^n_i$ donde las $k$ constantes
		 $c_1, c_2, c_3, .., c_k$ se pueden hallar sustituyendo $t_n$ por los casos base y 
		 hallando las soluciones del sistema resultante.
    }
    \item{
		 Cuando las ra\'ices no son todas diferentes, si $m_i$ es la multiplicidad de la ra\'iz $r_i$, 
		 entonces la soluci\'on tiene la foma $t_n=$ %arreglar
		donde todas las $c$, son potencialmente diferentes, y pueden ser indexadas como $c_1, c_2, .., c_k$
    }
    
\end{itemize}

\textbf{Rec. Lineal No Homogenea y Miembro Derecho de la Forma $b^n P(n)$}
\begin{itemize}
	\item{
		 La forma general es  $a_0 t_n + a_1 t_{n {−1}} + \cdots + a_k t_{n − k} = b^n P(n)$ donde $b$ es una 
		 constante y $P(n)$ es un polinomio de grado $d$.
	}
	\item{
		La ecuacion caracter\'istica es $a_0 x^k + a_1 x^{k {−1}} + \cdots + a^k = 0$
	}
\end{itemize}
