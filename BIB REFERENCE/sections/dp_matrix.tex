\begin{itemize}
    \item{
		Esta t\'ecnica puede ser usada en recurrencias de la forma
		$$ f(n, s_i) = \sum_{j=1}^k {c_{i, j} \cdot f(n-1, s_j)}$$ 
		donde $s_1, s_2, \cdots s_k $ son los $k$ estados donde puede escontrarse nuestro sistema
		y $n$ representa una condicion del estado adicional, que no influye en las transiciones entre 
		los $k$ estados $s_i$. 
    }
    \item{
		Para resolver tal recursi\'on definimos una matriz de trnasiciones $M$ donde cada elemento 
		$m_{i, j} = c_{i, j}$ o sea el ``peso" de la transici\'on desde el estado $i$ al $j$. 
    }
    \item{
		Ahora consid\'erese la matriz $M^{n-1}$. La suma de todos los elemetos el la fila $i$ nos dar\'a 
		$f(n, s_i)$, asumiendo que $ \forall i : f(0, i) = 0 $  
	}
	\item{
		Este proceso es similar a calcular la cantidad de caminos de largo $n$ en un grafo. Para ello elevamos la 
		matriz de adyacencia al exponente $n$ y el valor en la fila $i$ columna $j$ de la matriz resultante es el 
		n\'umero de caminos de largo $n$ desde el v\'ertice $i$ al $j$. 
    }
\end{itemize}
