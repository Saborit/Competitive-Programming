\begin{itemize}
	\item{\textbf{Conjetura de Goldbach:}
		 Todo n\'umero par $n \geq 4$ puede ser expresado como la suma de dos n\'umeros primos.
	}
	\item{
		Todo n\'umero natural $n$ puede representarse como la suma de n\'umeros distintos menores que $s$, 
		donde $s = \lceil \frac{\sqrt{1 + 8n} - 1}{2} \rceil$, o sea, $s$ es la base del menor n\'umero triangular 
		que es mayor o igual que $n$. Es m\'as, sea $k = (\sum_{i=1}^s i) - n$, tenemos que $k \in [1, s]$. 
		Entonces $n$ puede ser representado como la suma de todos los n\'umeros desde uno hasta $s$, excluyendo a $k$.
		
	}
	\item{\textbf{Teorema de Fermat sobre la representacion como suma de cuadrados:}
		Un n\'umero primo $p$ se puede expresar de la forma $p=a^2 + b^2$ si y solo si se puede expresar 
		de la forma $p = 4c + 1$
	}
	\item{
		Un n\'umero $n$ puede ser representado como la suma de dos cuadrados si
	    es el producto de dos n\'umeros que pueden ser expresados como la suma de dos cuadrados.
	}
	\item{
		Un entero $n$ puede ser expresado como suma de dos cuadrados si y solo si cada divisor primo de 
		$n$ de la forma $4k+3$ tiene exponente par en la representaci\'on can\'onica de $n$.
	}
	\item{
	    Si $p$  es un primo impar que divide a $a^2 + b^2$ con $gcd(a, b) = 1$, entonces $p \equiv 1 (mod \text{ } 4)$
	}
	\item{
		Cualquier entero positivo puede ser expresado como la suma de tres cuadrados enteros positivos
		si y solo si no es de la forma $4^n(8k+7)$, donde $n$ y $k$ son enteros no negativos.
	}
	\item{
		(Lagrange, 1770): Cada entero positivo puede ser representado como la suma de cuatro o menos cuadrados 
		enteros.
	}
\end{itemize}

