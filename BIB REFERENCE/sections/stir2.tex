\begin{itemize}
    \item{
		El n\'umero de Striling de segundo tipo
		$\left\{\begin{smallmatrix}
			n\\
			k
		\end{smallmatrix}\right\}$
		cuenta el n\'umero de formas de particionar un conjunto de $n$ objetos en $k$ subconjuntos no vac\'ios. 
		Hay solo una forma de poner $n$ elementos en un solo conjunto no vac\'io, y solo una forma de 
		distribuirlos en $n$ subconjuntos, por tanto, 
		$\left\{\begin{smallmatrix}
			n\\
			1
		\end{smallmatrix}\right\} 
		= 
		\left\{\begin{smallmatrix}
			n\\
			n
		\end{smallmatrix}\right\}
		= 1$
		para toda $n > 0$. 
		
		Por otro lado  
		$\left\{\begin{smallmatrix}
			0\\
			1
		\end{smallmatrix}\right\} = 0$ porque tal subconjunto est\'a vac\'io. 
		
		Por acuerdo se toma
		$\left\{\begin{smallmatrix}
			0\\
			0
		\end{smallmatrix}\right\} = 1$;
		y como un conjunto no vac\'io necesita al menos una parte
		$\left\{\begin{smallmatrix}
			n\\
			0
		\end{smallmatrix}\right\} = 0$ para todo $n>0$.	
    }
    \item{
       Dado un conjunto de $n > 0$ elementos para ser particionados en $k$ subconjuntos no vac\'ios de 
       $\left\{\begin{smallmatrix}
			n\\
			k
		\end{smallmatrix}\right\}$
	   maneras, podemos, o poner el \'ultimo objeto en un nuevo subconjunto formado por \'el mismo 
	   (esto se puede hacer de 
       $\left\{\begin{smallmatrix}
			n - 1\\
			k - 1
		\end{smallmatrix}\right\}$
	   formas), o lo colocamos en alguno de los conjuntos formados con los $n-1$ primeros objetos.  
	   Esto \'ultimo se puede hacer de 
	   $k
	   \left\{\begin{smallmatrix}
			n - 1\\
			k
		\end{smallmatrix}\right\}$  
		maneras, porque cada una de las 
		$\left\{\begin{smallmatrix}
			n-1\\
			k
		\end{smallmatrix}\right\}$
		formas de distribuir los $n-1$ objetos dan $k$ subconjunto en los que podemos colocar el n-\'esimo
		objeto.
		De ah\'i que:
		
		$$\left\{\begin{smallmatrix}
			n\\
			k
		\end{smallmatrix}\right\}
		= 
		k\left\{\begin{smallmatrix}
			n-1\\
			k
		\end{smallmatrix}\right\}
		+
		\left\{\begin{smallmatrix}
			n-1\\
			k-1
		\end{smallmatrix}\right\} 
		$$
    }
    \item{
        $\left[\begin{smallmatrix}
			n\\
			k
		\end{smallmatrix}\right]$
		representa igualmente el n\'umero de permutaciones de $n$ elementos que contienen exactamente
		$k$ ciclos. Por tanto
		$$
		\sum_{k=0}^n
		\left[\begin{smallmatrix}
			n\\
			k
		\end{smallmatrix}\right]
	    = n!	
		$$ para todo $n\geq0$ 
    }
    
\end{itemize}

