\begin{itemize}
    \item{
		\textbf{Teorema:}
		Sean $m_1, m_2, \cdots m_n $  enteros mayores que $1$ y primos relativos entre s\'i, y 
		$a_1, a_2, \cdots a_n$ enteros cualesquiera, entonces existe una soluci\'on $x$ para 
		las congruencias simult\'aneas
		
		\begin{equation*}
			\left \{
				\begin{aligned}
					x \equiv a_1 (mod\text{ } m_1)\\
					x \equiv a_2 (mod\text{ } m_2)\\
					\cdots\cdots\cdots\cdots\cdots\cdot\cdot\\
					x \equiv a_n (mod\text{ } m_n)\\	
				\end{aligned}
			\right.
		\end{equation*}
	    si $x$ y $x'$ son dos soluciones, estonces $x\equiv x' (mod\text{ } M)$ donde $M = m_1 m_2 \cdots m_n$ 	
    }
    \item{
		\textbf{M\'etodos con M\'odulos No Necesariamente Coprimos:}\\
		Para dos ecuaciones tenemos
		
		\begin{equation*}
			\left \{
				\begin{aligned}
					x \equiv a_1 (mod\text{ } m_1)\\
					x \equiv a_2 (mod\text{ } m_2)\\
				\end{aligned}
			\right.
		\end{equation*}
		
		Aqu\'i $m_1$ y $m_2$ no necesariamente son coprimos.\\
		Este sistema de congruencias implica que
		
		\begin{equation*}
			\left \{
				\begin{aligned}
					x = a_1 + m_1 k_1\\
					x = a_2 + m_2 k_2\\
				\end{aligned}
			\right.
		\end{equation*}
		 
		para unos enteros $k_1, k_2$. \\
		Igualando los miembros derechos de la ecuaciones, tenemos que 
		$$ a_1 + m_1 k_1 = a_2 + m_2 k_2$$ 
		$$ m_1(-k_1) + m_2 k_2 = a_1 - a_2$$
		
		Ya que conocemos los valores de $a_1, a_2, m_1$ y $m_2$, solo hay que resolver una ecuaci\'on
		diof\'antica. Sea $d = gcd(m_1, m_2)$, sabemos que divide el primer miembro de la ecuaci\'on, 
		as\'i que para que haya soluciones, debe dividir tambi\'en al segundo. \\
		Usando el Algoritmo Extendido de Euclides, podemos encontrar un par $(x', y')$ tal que 
		$m_1 x' + m_2 y' = d $ . Luego multiplicamos ambos miembros por $\frac{a_1-a_2}{d}$, obteniendo
		
		$$m_1 x' \frac{a_1-a_2}{d} + m_2 y' \frac{a_1-a_2}{d} = a_1 - a_2 $$ 
		
		Entonces $k_1 = -x' \frac{a_1-a_2}{d} $ y $ k_2 = y' \frac{a_1-a_2}{d} $ Ahora podemos sustituir 
		$k_1$ en el segundo sistema para obtener $$ x = a_1 + x' \frac{a_2-a_1}{d} m_1 $$ 		 
    }
    \item{
		\textbf{Infinitas Soluciones}\\
		Digamos que tenemos dos soluciones $x_1$ y $x_2$. Ahora tenemos que $x_1 \equiv a_1 (mod\text{ } m_1)$ 
		y que $x_2 \equiv a_1 (mod\text{ } m_1)$. Entonces por transitividad de congruencias tenemos que 
		$ x_1 \equiv x_2 (mod\text{ } m_1) $ y haciendo lo mismo para $m_2$, $ x_1 \equiv x_2 (mod\text{ } m_2) $. \\
		Esas dos congruencias son equivalentes a $ x_1 \equiv x_2 (mod\text{ } lcm(m_1, m_2)) $ as\'i que las 
		infinitas soluciones son $x = x_0 + k \cdot lcm(m_1, m_2)$ donde $x_0$ es la soluci\'on original del
		sistema y $k \in \mathbb{Z}$
		
    }
    \item{
		\textbf{Implementaci\'on:}\\
		Ya que los n\'umeros pueden llegar a ser un poco grandes, debemos hacer los c\'alculos m\'odulo $l = lcm(m_1, m_2)$.
		Entonces
		
		$$ x = \left(a_1 + \left( \left(x' \cdot \frac{a_1 - a_1}{d}\right) mod\text{ }l \right) \cdot m_1 \text{ } mod\text{ }l\right) mod\text{ }l  $$
		
		Si $m_1, m_2 \leq 10^9$, entonces $lcm(m_1, m_2) \leq 10^{18}$, $x' \leq 10^9$, $\frac{a_2 - a_1}{d} \leq 10^9$
		Podemos calcular $x' \cdot \frac{a_2 - a_1}{d} $, pero ya que e\'l y $l$ pueden ser hasta $10^{18}$, la multiplicaci\'on
		final por $m_1$ causar\'a un desbordamiento. Para manejar este problema usaremos la propiedad:
		$$ c a \text{ mod } cb = c( a \text{ mod 	} b )$$
		Sabiendo que $lcm(m_1, m_2) = \frac{m_2}{d} \cdot m_1$ y tomando en la propiedad anterior $a = x' \cdot \frac{a_2-a_1}{d}$, 
		$b = \frac{m_2}{d}$ y $c = m_1$, tenemos que 
		$\left( x' \cdot \frac{a_2 - a_1}{d} \text{ mod } l \right) \cdot m_1 \text{ mod } l$ es igual a:
		$$ \left( x' \cdot \frac{a_2 - a_1}{d} \text{ mod } \frac{m_2}{d} \right) \cdot m_1 \text{ mod }l$$
		Ya que $\frac{m_2}{d} \leq 10^9$, es problema de desbordamiento est\'a solucionado.
    }
    \item{
		\textbf{Con M\'as de Dos Relaciones de Congruencias:}\\
			Digamos que tenemos $t$ congruencias de la forma 
			$$ x \equiv a_i \text{ mod } m_i \text{, para } i = 1, 2, \cdots, t$$
			Entonces podemos mezclar las relaciones una por una. Luego de resolver el sistema formado por las primeras dos 
			relaciones, obtenemos una nueva de la forma 
			$$ x \equiv s \text{ mod } lcm(m_1, m_2) $$
			Ahora podemos mezclarla con la tercera, y as\'i sucesivamente... Este algoritmo tiene una complejidad de 
			$O(t \cdot \log lcm(m_1, m_2, \cdots, m_t))$ 
    }
\end{itemize}
