\begin{itemize}
    \item{
		\textbf{Definici\'on:} 
		Dado un \'arbol con $n$ nodos, un centroide es un nodo cuya eliminaci\'on separa al \'arbol 
		dado en un bosque, donde cada \'arbol resultante no contiene m\'as de $n/2$ v\'ertices. Para 
		un \'arbol cualquiera siempre existe al menos un centroide.
    }
    
    \item{
		\textbf{Propiedades de los Centroides}
		\begin{itemize}
			\item{
				Si llamamos peso de un nodo al n\'umero de v\'ertices en el mayor sub\'arbol de dicho
				nodo, un nodo con m\'inimo peso es un centroide del \'arbol. 
			}
			\item{
				Cada \'arbol tiene uno o dos centroides. Si un \'arbol tiene dos centroides, entonces 
				estos est\'an unidos por una arista y la aliminaci\'on de dicha arista deja dos \'arboles
				de igual cantidad de nodos.
			}
			\item{
				Un v\'ertice es el \'unico centroide de un \'arbol de $n$ v\'ertices si y solo si su peso 
				es a lo m\'as $\frac{n-1}{2}$. 
			}
			\item{
				Definimos $d(u, v)$ como el largo del camino m\'as corto entre $u$ y $v$. La distancia del 
				v\'ertice $u$, denotado $L(u)$, es la suma de todas las $d(u, v)$ para cada $v$, v\'ertice 
				del \'arbol.
				$$ L(u) = \sum_{v \in T} d(u, v) $$
				Un centroide del \'arbol es tal que $L(u)$ es m\'inimo.   
		    }
			\item{
				Si modelamos con el \'arbol, un sistema de l\'ineas telef\'onicas, donde cada arista representa
				un sector de l\'inea y los caminos llamadas entre los v\'ertices finales. Se asume que, al mismo 
				tiempo, un v\'ertice puede estar envuelto en solo una llamada. \\ 
				Definimos el ``switchboard number"  del v\'ertice $v$, $sb(v)$, como el m\'aximo n\'umero de llamadas
				a trav\'es de $v$ en un momento determinado. Los centroides de del \'arbol en tal modelo, cumplen 
				que $sb(v)$ es m\'aximo.    
			}
		\end{itemize}
		
		\item{
			\textbf{Encontrar el Centroide de un \'Arbol:}
              Una manera de encontrar el centroide es elegir una ra\'iz arbitraria, entonces hacer un BFS calculando 
              la cantidad de nodos en cada sub\'arbol. Luego nos movemos, comenzando desde la ra\'iz, al sub\'arbol
              de mayor tama\~no hasta que lleguemos a un v\'ertice cuyo sub\'arbol no tenga m\'as de $n/2$ nodos. Este
              nodo es el centroide del \'arbol.
		}
		
		\item{
			\textbf{Descomponiendo el \'Arbol para Formar un \'Arbol de Centroides:}
			Cuando se elimina el centroide, el \'arbol original de descompone en un n\'umero de \'arboles
			diferentes, cada uno con menos de $n/2$ nodos. Tomamos este centroide como la ra\'iz de un nuevo 
			\'arbol al que llamaremos ``centroid tree" o \'arbol de centroides. Entonces hacemos lo mismo 
			recursivamente para los \'arboles que se formaron e incluimos los centroides de estos como hijos
			de nuestra ra\'iz. La recursi\'on para cuando el \'arbol est\'a formado por un solo v\'ertice.
			As\'i un nuevo centroid tree se forma a partir de nuestro \'arbol original.   
		}
		\item{
			\textbf{Propiedades de los \'Arboles de Centroides}
			\begin{itemize}
				\item{
					El \'abol formado contiene los $n$ nodos del \'arbol original.
				}
				\item{
					La altura del \'arbol de centroides es $O(\log n)$ 
				}
				\item{
					Consider any two arbitrary vertices A and B and the path between them (in the original tree) 
					can be broken down into A-->C and C-->B where C is LCA of A and B in the centroid tree.
					
					Consid\'erese cualesquiera dos v\'ertices $u$ y $v$ y el camino entre ellos en el \'arbol original.
					Dicho camino $ u \rightarrow v $, puede ser dividido en dos caminos $ u \rightarrow lca(u, v) $
					y $ lca(u, v) \rightarrow v $, donde $ lca(u, v) $ es el ancestro com\'un m\'as bajo entre $u$ y $v$
					en el centroid tree.   
				}
				\item{
					Se descompone el \'arbol original en $O(n \log n )$ caminos diferentes (desde cada centroide 
					a todos los vertices en su sub\'arbol en el centroid tree) tales que cualquier camino en el \'arbol 
					original es la concatenaci\'on de dos caminos diferentes de este conjunto.  
			   }
			\end{itemize}
		}
    }
\end{itemize}
