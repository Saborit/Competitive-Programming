\begin{itemize}
    \item{
		\textbf{Estructura:}
		El Aho-Corasick est\'a compuesto por las siguientes tres funciones:
		\begin{itemize}
			\item{
				\textbf{Go To:}
				La funci\'on $g(s, c)$ que simplemente sigue las aristas en un trie formado por todos los 
				patrones dados. El nodo $u$ es el padre de $g(u, c)$ en el trie y la arista entre ellos 
				est\'a etiquetada con el caracter $c$.
			}
			\item{
				\textbf{Failure:}
				Esta funci\'on $f(s)$ guarda todas las aristas correspondientes a cuando el caracter actual 
				no tiene una arista en el \'arbol.
		    }
		    \item{
				\textbf{Salida:}
				La funci\'on $o(s)$ que guarda los \'indices de todos patrones al final del estado actual. 
		    }
		\end{itemize} 
    }
    \item{
		\textbf{Preprocesamiento:}
		
		\begin{itemize}
			\item{
				Primero se construye un trie con todas los patrones (funci\'on $g$). 
			}
			\item{
			    Luego construimos la funci\'on de fallo $f$. Para un estado encontramos el sufijo propio 
			    m\'as largo que es tambi\'en prefijo de alg\'un patr\'on, o sea, que tiene un nodo
			    dentro del trie que lo representa. Esto se hace con un BFS. 
			}
			\item{
				Para calcular $f(u)$ se busca el primer se hace $w$ tal que $g(w, )$ 	
			}
			\item{
				La funci\'on de salida $o(u) = o(u) \cup o(f(u))$. Esto se hace porque los patrones reconocidos 
				en $f(u)$, y solo esos, son sufijos propios de la cadena que representa el nodo $u$, y ser\'an
				entonces reconocidos tambi\'en en el estado $u$.  
			}
			\item{
				\textbf{}
				
			}
			\item{
				\textbf{}
				
			}
		\end{itemize} 
    }
\end{itemize}
