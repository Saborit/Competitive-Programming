\begin{itemize}
    \item{
		\textbf{Propiedades de los Juegos Imparciales}
		\begin{itemize}
			\item{
				Hay dos jugadores y el conjunto de posiciones es finito.
			}
			\item{
				Ambos jugadores alternan turnos y el conjunto de jugadas posibles depende de la posici\'on 
				y no de que jugador est\'a en turno.
			}
			\item{
				Si es un juego normal (normal play), el juego termina cuando uno de los jugadores no puede 
				hacer un movimiento, el jugador que no puede mover pierde; si se juega "misere", el que no 
				puede mover gana. 
			}
		\end{itemize}
    }
    
    \item{
		\textbf{Posiciones en los Juegos Imparciales}
		\begin{itemize}
			\item{
				El juego est\'a en una P-posici\'on si se asegura una victoria para el jugador $P$revio.
			}
			\item{
				El juego est\'a en una N-Posici\'on si se asegura una victoria para el pr\'oximo jugador($N$ext).
			}
			\item{
				Una posici\'on $A$ es seguidora de otra posici\'on $B$ si el jugador en turno puede ir de $B$ a $A$ 
				con su movimiento. Al conjunto de las posiciones seguidoras de $A$ se las denota $F(A)$.
			}
			\item{
				Una posici\'on es terminal si no tiene ninguna seguidora, o sea el juego termina si se llega a ella.	
			}
		\end{itemize}
    }
    
    \item{
		\textbf{Inducci\'on Hacia Atr\'as}
		\begin{itemize}
			\item{
				Etiquete cada posici\'on terminal como $P$.
			}
			\item{
				Etiquete cada posici\'on que tiene como seguidora a una $P$ como $N$.
			}
			\item{
				Etiquete cada posici\'on que solo tiene como seguidoras a N-posiciones como P-posiciones.
			}
			\item{
				Si se juega misere, en el paso $1$, cada posicion terminal es $N$.
			}
		\end{itemize}
    }
     
    \item{
		\textbf{Nim y Suma Nim}
		\begin{itemize}
			\item{
				El Nim es un juego donde se cuenta con $n$ montones de fichas. Dos jugadores alternan movimientos 
				donde en cada uno pueden tomar un n\'umero de fichas de un mont\'on. Quien no pueda tomar una ficha 
				(cuando no quede ninguna), pierde.
			}
			\item{
				La suma nim de uno o varios n\'umeros es simplemente el $xor$ de estos. 
			}
			\item{
				Cada juego combinatorio imparcial se comporta como un juego de nim.
			}
			\item{
				En nim, una posici\'on es $P$ si y solo si la suma nim de los mont\'iculos es igual a cero.
			}
		\end{itemize}
    }
    
    \item{
		\textbf{Poker Nim y Las Jugadas Reversibles}
		\begin{itemize}
			\item{
				El Poker Nim es esencialmente el mismo juego del nim pero tambi\'en se puede adicionar fichas a los mot\'iculos. 
			}
			\item{
				Cada vez que el oponente adicione fichas, el jugador puede sustraer la misma cantidad. Es una jugada reversible.
			}
		\end{itemize}
    }
    
    \item{
		\textbf{Juegos en Grafos Dirigidos}
		\begin{itemize}
			\item{
				Un juego puede verse como un grafo dirigido donde los nodos son las posiciones y las aristas las jugadas.
			}
			\item{
				Desde cada nodo, hay una arista a cada uno de los nodos correspondientes a sus seguidores.			
			}
		\end{itemize}
    }
    \item{
		\textbf{Funci\'on Sprage-Grundy}
		\begin{itemize}
			\item{
				Es una funci\'on $G$ que le asigna a cada nodo del grafo del juego, un n\'umero de tal forma que:
				$$ G(x) = min \left\{ n \geq 0 \colon n \neq G(y), \forall y \in F(x)  \right\} $$
				o sea, $G(x)$ es el menor entero que no figura entre los valores Grundy correspondientes a las 
				posiciones seguidoras de $x$.
			}
			\item{
				El valor de la funci\'on Grundy para las posiciones terminales es cero.
			}
			\item{
				Una posici\'on es perdedora para el jugador en turno si y solo si el valor Grundy es cero.
			}
		\end{itemize}
    }
    \item{
		\textbf{Suma Disyuntiva de Juegos Combinatorios}\\
				Dados varios juegos combinatorios, se puede formar un nuevo juego combinando estos, jugando 
				bajo las siguientes reglas:
		\begin{itemize}
			\item{
				Se da una posicion inicial para cada uno de los juegos.
			}
			\item{
				Los jugadores alternan jugadas. Una jugada consiste en seleccionar uno o varios de los subjuegos y 
				hacer una jugada legal en ellos, dejando igual los dem\'as. 
			}
			\item{
				Se llega a una posici\'on terminal cuando no se puede hacer una jugada en ninguno de los subjuegos.  
			}
		\end{itemize}
    }
    \item{
		\textbf{Reglas Generales de la Suma de Juegos}
		\begin{itemize}
			\item{
				\textbf{Si el jugador en turno elige un juego y hace una jugada en este}: Se aplica el Teorema Sprague-Grundy.
			}
			\item{
				\textbf{Si el jugador en turno elige un conjunto no vac\'io de juegos (posiblemente todos) y hace una jugada 
				en todos ellos}: Una posici\'on es perdedora si y solo si cada juego est\'a en una posici\'on perdedora.
			}
			\item{
				\textbf{Si el jugador en turno elige un subconjunto propio (no vac\'io, pero no todos) de juegos y hace una jugada 
				en todos ellos}: La posici\'on es perdedora ssi todos los valores Grundy de los juegos son iguales.
			}
			\item{
				\textbf{Si el jugador debe jugar en todos los juegos y pierde si no puede mover en alguno}: Una posici\'on es perdedora 
				ssi alguno de los juegos esta en una posici\'on perdedora.
			}
		\end{itemize}
    }
    \item{
		\textbf{Propiedades de la Suma de Juegos}
		\begin{itemize}
			\item{
				 \textbf{Teorema Sprague-Grundy}: Si $G_i$ es la funci\'on Grundy para cada 
				 grafo $i$, con $1 \leq i \leq n $, entonces el grafo de la suma de los $n$ juegos 
				 tiene funci\'on Sprague-Grundy $G$ tal que 
				 $$ G(x) = G_1(x) \oplus G_2(x) \oplus \cdots \oplus G_n(x) $$ 
				 o sea, la suma nim de las funciones.
			}
			\item{
				\textbf{Producto Paralelo}: Si el jugador en turno debe hacer una jugada legal en 
				cada uno de los subjuegos, el valor de la funci\'on Grundy es el m\'inimo de las 
				funciones Grundy de cada posici\'on de los juegos componentes. 
			}
			\item{
				\textbf{Nim de Moore}: La suma de orden $p$ de varios juegos, es la suma disyuntiva 
				en la cual un jugador debe hacer una jugada legal en al menos uno de los subjuegos 
				y en a lo mas $p$ juegos. La suma usual es de orden $1$. La suma de orden $p$ se calcula 
				tomando la extensi\'on binaria de los tama\~nos. Se calculan los valores $w_l$  como
				$$ w_l = ((n_1)_l + (n_2)_l + \cdots + (n_k)_l) mod (p+1) $$
				donde $(n_i)_l$ es el bit l-\'esimo del i-\'esimo tama\~no de un mont\'iculo. 
			}
		\end{itemize}
    }
    \item{
		\textbf{El Nim de Lasker y los Juegos Take-and-Break}
		\begin{itemize}
			\item{
				En el Nim de Emanuel Lasker, los jugadores tienen otra posible jugada: dividir una pila 
				en dos pilas no vac\'ias, sin quitar ninguna ficha del juego.
			}
			\item{
				Se calcula la funci\'on Sprague-Grundy para este juego pero teniendo en cuenta que al hacer 
				una jugada en un mont\'iculo, las posiciones seguidoras de la actual, ser\'an no solamente 
				las correspondientes a los mont\'iculos menores, sino tambi\'en las de dividir un mot\'iculo 
				en dos.
			}
			\item{
				Por ejemplo $G(0) = 0$ y $G(1) = 1$. Los seguidores de $2$ son $0$, $1$, y $(1, 1)$; sus valores 
				Grundy son $0$, $1$, y el valor de la suma nim de los valores Grundy: $G(1) \oplus G(1) = 0$. 
				El valor Grundy es $G(2) = 2$. La posicion $3$ tiene como seguidores a $0$, $1$, $2$ y $(1, 2)$; 
				$G(3) = 4$, mientras que la $4$ tiene se seguidores a $0$, $1$, $2$, $(1, 3)$, $(2, 2)$. $G(4) = 3$. 
				Este patron se repite en todo el grafo.       
			}
		\end{itemize}
    }
\end{itemize}
