\begin{itemize}
    \item{
		$ \sum_{i=1}^{n} {i} = \frac{n(n+1)}{2} $	
    }
    \item{
		$ \sum_{i=1}^{n} i^2 = \frac{n(n+1)(2n+1)}{6} $
    }
    \item{
		$ \sum_{i=1}^{n} i^3 = \left( \sum_{i=1}^{n} i \right) ^ 2 = \left( \frac{n(n+1)}{2} \right) ^ 2 $
    }
    \item{
		$ \sum_{i=1}^{n} i^4 = \frac{6n^5 + 15n^4 + 10n^3 - n}{30}$
    }
    \item{
		$ \sum_{i=1}^{n} i^5 = \frac{2n^6 + 6n^5 + 5n^4 - n^2}{12} $
    }
    \item{
		$ \sum_{i=1}^{n} i_{par} = n(n+1) $
    }
    \item{
		$ \sum_{i=1}^{n} i_{impar} = \left( \frac{n+1}{2} \right) ^ 2$
    }
    \item{
		$ \sum_{i=1}^{n} r^i = \frac{r^{n+1} - 1}{r - 1} $
    }
    \item{
		$ \sum_{i=1}^{n} \frac{1}{i^k} = H_n^k $ donde $H_n^k$ es un n\'umero arm\'onico generalizado 
    }
    \item{
        Las siguientes son dos f\'ormulas para la suma de potencias con igual exponente y diferente base:
		$$ \sum_{i=1}^{n} i^p = \frac{(n+1)^{p+1}}{p+1} + \sum_{k=1}^{p} {\frac{B_k}{p-k+1}} C^p_k (n+1)^{p-k+1} $$
		donde 
		$ B_k = - \sum_{i=0}^{k-1} C^k_i \frac{B_i}{k+1-i} $ y $B_0 = 1$
		
		Esta otra usa n\'umeros de Stirling de segundo tipo:
		$$ \sum_{i=1}^n i^p = \sum_{r=1}^p 
		\left\{ \begin{array}{crl}
			p \\ 
			r
        \end{array} \right\}
		r! \binom{n+1}{r+1}
		$$
    }
    \item{
		Familia de f\'ormulas de suma $i a^i$:	
		$$ \sum_{i=0}^{n-1} i a^i = \frac{a - n a^n + a^{n+1} (n-1)}{(1 - a)^ 2} $$
		$$ \sum_{i=0}^{n-1} i 2^i = 2^n (n-2) + 2 $$ que es un caso especial para $a = 2$
    }
    \item{
		Propiedades de los logaritmos y las potencias generalizadas:
		$$ \sum_{i=s}^{t} \ln f(n) = \ln \prod_{n=s}^t f(n) $$
		$$ c^{\left( \sum_{i=s}^{t} f(n)\right)} = \prod_{n=s}^{t} c^{f(n)} $$
    }
\end{itemize}
