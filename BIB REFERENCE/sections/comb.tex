\begin{itemize}
    \item{
		\textbf{Variaciones sin Repeticion:}
		Se dan $n$ objetos. Se forman todas las distribuciones ordenadas de los $k$ objetos. 
		El n\'umero de distribuciones es $$V_k^n = \frac{n!}{(n-k)!}$$
    }
    \item{
		\textbf{Variaciones con Repetici\'on:}
		Se dan objetos que pertenecen a $n$ formas distintas. Se forman todas las
		distribuciones ordenadas de $k$ objetos en las cuales pueden figurar
		objetos repetidos. Dos distribuciones se consideran iguales si en cada
		posici\'on tienen objetos iguales respectivamente. El nu\'mero de estas
        distribuciones es $${\overline{V}}_k^n = n^k$$
    }
    \item{
		\textbf{Permutaciones sin Repetici\'on:}
			Se dan $n$ objetos. Se forman todas las distribuciones ordenadas de los $n$
			objetos. El n\'umero de distribuciones es $$P_n = n!$$
    }
    \item{
		\textbf{Permutaciones con Repetici\'on:}
			Se dan $n$ objetos que pertenecen a $k$ clases. Se forman todas las distribuciones 
			ordenadas de los $n$ objetos. Los objetos de una misma clase son iguales y su cantidad 
			es $n_1, n_2, \cdots, n_k$ respectivamente, de tal forma que $n_1 + n_2+ \cdots + n_k = n$. 
			El n\'umero de distribuciones es
			$$ P_(n_1, n_2, \cdots n_k) = \frac{n!}{n_1! n_2! \cdots n_k!} $$
    }
    \item{
		\textbf{Combinaciones sin Repetici\'on:}
		Se dan $n$ objetos y se quieren separar en dos grupos, uno de $k$ objetos y el otro de $n-k$. 
		El n\'umero de maneras de hacerlo es
		$$ C^n_k = \frac{n!}{k! (n-k)!} $$ 
    }
    \item{
		\textbf{Combinaciones con Repetici\'on:}
		Se tienen objetos de $n$ tipos diferentes. De cu\'antas formas podemos tomar $k$ objetos sin importar 
		que tomemos objetos de un mismo tipo.
		$$ \overline{C}_k^n = C_k^{n+k-1} $$
    }
\end{itemize}
