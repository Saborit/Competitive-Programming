\textbf{Definici\'on}
\begin{itemize}
	\item{ es un v\'ertice tal, que si se elimina junto a todas sus aristas adyacentes, causa que el 
		grafo se desconecte. }
	\item{ un grafo que no tiene puntos de articulaci\'on es una componente biconexa}
	\item{ si un nodo $v$ tiene un decendiente en el \'arbol del DFS, del que sale un back edge hacia 
		un ancestro de $v$ (o sea, se forma un ciclo), este nodo no puede ser un punto de articulaci\'on, 
		porque no es obligatorio pasar por \'el para llegar a sus hijos .
	}
\end{itemize}

\textbf{Algoritmo Hopcroft-Tarjan que Encuentra Puntos de Articulaci\'on}
\begin{itemize}
	\item{ 
		a medida que corremos el DFS, llenamos dos arreglos que nos ayudar\'an a clasificar las aristas: $dt[]$ y
		$low[]$ . $dt[u]$ guarda el n\'umero de orden del nodo $u$ en el DFS y $low[u]$ guarda el menor $dt[]$ de un nodo en el
		sub\'arbol del DFS de $u$, al que se puede llegar desde $u$. 
	}
	\item{
		Al principio $low[u] = dt[u]$. Entonces si hay un ciclo, o sea un back edge, $low[u]$ se puede hacer m\'as peque\'no. 
		N\'otese que si hay una arista hacia atr\'as (back edge) hacia el padre $p_u$ de $u$ no se actualiza $low[u]$ con $dt[pu]$. 
		O sea que los ciclos que cuentan son los que tienen $3$ aristas o m\'as.
    }
    \item{
		Cuando se est\'a en el v\'ertice $u$ que tiene a $v$ como un v\'ertice adyacente y $low[v] \geq dt[u]$ entonces $u$ es un
		punto de articulaci\'on. Esto es porque el hecho de que $low[v]$ no es menor que el $dt[u]$ implica que no hay
		back edge desde $v$ que pueda alcanzar a otro v\'ertice $w$ con menor $dt[w]$ que $dt[u]$. Un nodo $w$ con
		menor $dt[w]$ que $u$ implica que $w$ es ancestro de $u$ en el árbol del DFS. Esto significa que para llegar a un ancestro(s) 
		de $u$ desde $v$ hay que pasar por el v\'ertice $u$. Entonces, quitarlo causa que el grafo se desconecte.
    }
    \item{
		Caso especial: la ra\'iz del \'arbol del DFS, es un punto de articulaci\'on si y solo si tiene m\'as de un hijo.
    }
\end{itemize}

